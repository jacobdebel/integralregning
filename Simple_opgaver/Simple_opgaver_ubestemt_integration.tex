% Intended LaTeX compiler: xelatex
\documentclass[a4paper, 12pt]{article}
\usepackage{graphicx}
\usepackage{grffile}
\usepackage{longtable}
\usepackage{wrapfig}
\usepackage{rotating}
\usepackage[normalem]{ulem}
\usepackage{amsmath}
\usepackage{textcomp}
\usepackage{amssymb}
\usepackage{capt-of}
\usepackage{hyperref}
\usepackage[danish]{babel}
\usepackage{mathtools}
\usepackage[margin=3.0cm]{geometry}
\hypersetup{colorlinks, linkcolor=black, urlcolor=blue}
\setlength{\parindent}{0em}
\parskip 1.5ex
\author{Matematik A}
\date{Vibenshus Gymnasium}
\title{Simple opgaver\\\medskip
\large Ubestemt integration}
\hypersetup{
 pdfauthor={Matematik A},
 pdftitle={Simple opgaver},
 pdfkeywords={},
 pdfsubject={},
 pdfcreator={Emacs 27.1 (Org mode 9.3)}, 
 pdflang={Danish}}
\begin{document}

\maketitle

\section*{Introduktion}
\label{sec:org5ec3a09}

Alle opgaver skal besvares uden brug af CAS. Brug i stedet den udleverede formelsamling til at slå stamfunktioner op for udvalgte funktionstyper.

Husk, at der ikke er noget som hedder kæderegel, produktregel eller kvotientregel for integration! Vi vil se på metoder senere, som i \emph{nogle} tilfælde kan bruges, men ikke altid.

\section*{Opgave 1}
\label{sec:orgecebe7a}

Bestem følgende ubestemte integraler uden brug af CAS. Nogen gange skal integranden (funktionen, som skal integreres) omskrives, \emph{inden} selve integrationen finder sted.

\begin{enumerate}
\item \(\int x -1 \,dx\)
\item \(\int x^2 -1 \,dx\)
\item \(\int x^{-2} -1 \,dx\)
\item \(\int \frac{1}{x^3} \,dx\)
\item \(\int \frac{1-x}{x^3} \,dx\)
\item \(\int \frac{2-x}{x-2}\,dx\)

\newpage
\end{enumerate}
\section*{Opgave 2}
\label{sec:org149869f}
Bestem, uden brug af CAS, stamfunktionerne til hver af følgende funktioner:

\begin{align*}
f(x) &= - \sin \left( x \right) \\
g(x) &= - \cos \left( x \right) +7 \\
h(x) &= - 2 \cdot \cos \left( x \right) \\
d(x) &= - \cos \left( x \right) + 2 \cdot \sin \left( x \right)\\
e(x) &= x^4 -x \\
t(x) &= x^{\frac{2}{5}} - \sqrt{x} \\
m(x) &= \frac{1}{x^{3.1}} \\
n(x) &= e^x - x^e 
\end{align*}
\end{document}